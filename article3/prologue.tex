\chapter{Pr\'{e}ambule au Troisi\`{e}me Article }

Les deux premiers articles résument nos approches pour des problématiques
différentes: des entrées où aucun a priori n'est disponible ou au contraire,
des données dont la structure peut être utilisée à notre avantage. Alors que
les précédents travaux focalisent sur l'aspect technique de différents
algorithmes d'apprentissage, ce troisième article s'intéresse à nos intuitions
concernant la structure de l'espace sémantique de représentation appris. En particulier, nous
nous intéressons à la capacité de visiter des classes différentes lors du
processus d'échantillonnage à partir de l'espace de représentation sémantique.

\section{D\'{e}tails de l'article}

{\bf Better Mixing via Deep Representations} Yoshua Bengio\footnote{indique une contribution similaire}, Grégoire Mesnil$^{1}$, Yann Dauphin and Salah Rifai. {\it
International Conference on Machine Learning} $2013$\\

{\it Contribution Personnelle} Le point de départ de cette idée a été présenté
à Snowbird à un workshop \citep{Mesnil-et-al-LW2012}. Cela consistait à montrer
sur support vidéo comment il était possible de se déplacer d'un exemple à un
autre dans l'espace des représentations tout en restant sur la variété des
exemples d'apprentissage. Cela a renforcé les intuitions de Yoshua concernant
un problème de mixing entre différents modes et pourquoi il était plus aisé
de visiter toutes les classes lors du processus d'échantillonnage dans l'espace
abstrait des représentations. À partir de modèles de RBMs et de CAEs entraînés
par Salah Rifai et Yann Dauphin, j'ai réalisé et conçu l'intégralité des
expériences visant à vérifier les hypothèses de Yoshua, qui lui a rédigé
l'article. Il a été décidé ensemble que nous avions une contribution égale pour
cet article.

\section{Contexte}

On a toujours peu d'intuitions sur ce que les représentations apprises par les
réseaux de neurones contiennent exactement ou sur la structure particulière de
l'espace des représentations. Cette publication visait à présenter et valider nos intuitions:

\begin{center}
\framebox{
\begin{minipage}{0.8\linewidth}
{\bf Hypothèse 1: Profondeur et Mixage durant l'échantillonnage.} 

Une architecture profonde apprend un espace sémantique dans lequel les chaînes
de Markov mixent plus rapidement entre différentes classes.  

 \end{minipage}}
 \end{center}

\begin{center}
\framebox{
\begin{minipage}{0.8\linewidth}
{\bf Hypothèse 2: L'espace sémantique appris rend la variété initiale localement linéaire.}

 La variété des exemples d'apprentissage complexe et non linéaire dans l'espace
 initial est rendue localement linéaire dans l'espace sémantique défini par les
 couches abstraites. Le volume de l'espace sémantique est occupé de façon plus
 uniforme que l'espace d'entrée.

\end{minipage}
}\\
\end{center} 

\section{Contributions}

Ce travail a été cité par plusieurs articles
inspirés par cette hypothèse de remplissage de l'espace de représentation et de
structure de la variété des exemples d'apprentissage. 

\section{R\'{e}cents d\'{e}veloppements}

Ces idées sont toujours d'actualité et restent dans l'esprit des chercheurs
pour guider leurs intuitions. Le concept de naviguer sur la variété dans l'espace des
représentations a notamment été repris par Ian
Goodfellow\footnote{https://plus.google.com/+IanGoodfellow/posts/SJxfk4SeNi7}
pour présenter la variété des chiffres appris par les Adversarial Networks
\citep{Goodfellow-et-al-ARXIV2014}.
