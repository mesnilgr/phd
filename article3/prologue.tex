\chapter{Pr\'{e}ambule au Troisi\`{e}me Article }

\section{D\'{e}tails de l'article}

{\bf Better Mixing via Deep Representations} Yoshua Bengio\footnote{Indicates
equal contribution}, Grégoire Mesnil$^{1}$, Yann Dauphin and Salah Rifai. {\it
International Conference on Machine Learning} $2013$\\

{\it Contribution Personnelle} Le point de départ de cette idée a été présenté
à Snowbird à un workshop \citep{Mesnil-et-al-LW2012}. Cela consistait à montrer
sur support vidéo comment il était possible de se déplacer d'un exemple à un
autre dans l'espace des représentations tout en restant sur la variété des
exemples d'apprentissage. Cela a renforcé les intuitions de Yoshua concernant
un problème de mixing entre différents modes et pourquoi cela était plus aisé
de visiter toutes les classes lors du processus d'échantillonage dans l'espace
abstrait des représentations. À partir de modèles de RBMs et de CAEs entraînés
par Salah Rifai et Yann Dauphin, j'ai réalisé et conçu l'intégralité des
exprériences visant à vérifier les hypothèses de Yoshua, qui lui a rédigé
l'article. Il a été décidé ensemble que nous avions une contribution égale pour
cet article.

\section{Contexte}

On a toujours peu d'intuitions sur ce que les représentations apprises par les
réseaux de neurones contiennent exactement ou la structure particulière de
l'espace des représentations. Cette publication visait à présenter et valider nos intuitions:

\begin{center}
\framebox{
\begin{minipage}{0.8\linewidth}
{\bf Hypothèse 1: Profondeur et Mixage durant l'échantillonage.} 

Une architecture profonde apprend un espace sémantique dans lequel les chaînes
de Markow mixent plus rapidement entre différentes classes.  

 \end{minipage}}
 \end{center}

\begin{center}
\framebox{
\begin{minipage}{0.8\linewidth}
{\bf Hypothèse 2: L'espace sémantique appris rend la variété localement linéaire.}

 La variété des exemples d'apprentissage complexe et non-linéaire dans l'espace
 initial est rendue localement linéaire dans l'espace sémantique défini par les
 couches abstraites. Le volume de l'espace sémantique est occupé de façon plus
 uniforme que l'espace d'entrée.

\end{minipage}
}\\
\end{center} 

\section{Contributions}

Les conclusions apportées par ce travail ont inspiré plusieurs travaux
utilisant cette hypothèse de remplissage de l'espace de représentation et de
structure de la variété des exemples d'apprentissage. Aussi, l'entraînement de
RBMs qui nécessite un échantillonage pendant la phase d'apprentissage, il est
désormais justifié d'effectuer le processus d'échantillonage à partir des
couches supérieures plutôt que dans l'espace d'entrée.

\section{R\'{e}cents d\'{e}veloppements}

Ces idées sont toujours d'actualité et restent dans l'esprit des chercheurs
pour guider leurs intuitions. Le concept de naviguer dans l'espace des
représentations a notamment été repris par le papier Adversarial Sampling
\citep{Goodfellow-et-al-ARXIV2014}.
