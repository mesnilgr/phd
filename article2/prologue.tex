\chapter{Pr\'{e}ambule au Deuxi\`{e}me Article }

\section{D\'{e}tails de l'article}

{ \bf Unsupervised Learning of Semantics of Object Detections for Scene
Categorization} Gr\'{e}goire Mesnil, Salah Rifai, Antoine Bordes, Xavier
Glorot, Yoshua Bengio and Pascal Vincent.  { \it Advances in Intelligent
Systems, Computing} Springer, Vol.  318, Maria De Marsico and Ana Fred (Eds):
Pattern Recognition Applications and Methods, 2015\\ 

{\it Contribution Personnelle} Le d\'{e}but de ce travail a \'{e}t\'{e}
effectu\'{e} en collaboration avec Antoine Bordes, Xavier Glorot et Salah
Rifai. L'id\'{e}e initiale \'{e}tait d'utiliser le contexte dans des images
(pr\'{e}sence d'objets) pour am\'{e}liorer la cat\'{e}gorisation de sc\`{e}nes.
Je me suis charg\'{e} de pr\'{e}parer les repr\'{e}sentations d'Object Bank
pour tous nos datasets et effectu\'{e} l'int\'{e}gralit\'{e} des
exp\'{e}riences except\'{e} celles relatives aux CAEs r\'{e}alis\'{e}es par
Salah Rifai. J'ai ensuite adapt\'{e} le pipeline de la comp\'{e}tition afin de
r\'{e}duire la dimension d'entr\'{e}e et permettre \`{a} l'apprentissage non
supervis\'{e} d'avoir lieu.  Apr\`{e}s plusieurs discussions fructueuses lors
d'une pr\'{e}sentation \`{a} un workshop de NIPS en 2011
\citep{Mesnil-workshop-nips}, l'id\'{e}e s'est am\'{e}lior\'{e}e et notre
publication \citep{Mesnil-icpram} a \'{e}t\'{e} accept\'{e}e \`{a} ICPRAM.
Apr\`{e}s l'oral, le papier a \'{e}t\'{e} selectionn\'{e} pour publication dans
un recueil contenant les meilleurs papiers pr\'{e}sent\'{e}s \`{a} cette
conf\'{e}rence \citep{Mesnil-icpram-journal}. J'ai r\'{e}dig\'{e} l'article en
collaboration avec les co-auteurs.

\section{Contexte}

\`{A} cette \'{e}poque les r\'{e}seaux de neurones convolutionnels n'ont pas
encore remport\'{e} la comp\'{e}tition ImageNet \citep{Krizhevsky-2012-small} et les
d\'{e}tecteurs d'objet \`{a} l'\'{e}tat de l'art restent ceux utilisés dans par
Object Bank \citep{lsvm-pami}. Pour la classification de sc\`{e}nes, Object
Bank est encore la m\'{e}thode \`{a} l'\'{e}tat de l'art.  C'est pourquoi nous
avons d\'{e}cid\'{e} d'utiliser les repr\'{e}sentations d'Object Bank comme
support \`{a} notre travail afin d'explorer si une combinaison des
d\'{e}tections d'objets de mani\`{e}re plus abstraite permettait un gain en
performance.

\section{Contributions}

Malgre ses tr\`{e}s bonnes performances \`{a} l'\'{e}poque, un des
inconv\'{e}nients d'Object Bank reste la dimensionnalit\'{e} tr\`{e}s
\'{e}lev\'{e}e de la repr\'{e}sentation.  Dans notre travail, nous avons
explor\'{e} diff\'{e}rentes techniques pour combiner les d\'{e}tections
d'objets afin de permettre aux chercheurs int\'{e}ress\'{e}s d'utiliser une
repr\'{e}sentation d'Object Bank plus compacte et plus performante.


\section{R\'{e}cents d\'{e}veloppements}

Il est certain qu'utiliser des d\'{e}tecteurs d'objets \`{a} base de
r\'{e}seaux de neurones convolutionnels am\'{e}liorerait nettement les
performances de notre approche \citep{Scene-14}.  Au moment du
d\'{e}veloppement de ces id\'{e}es, notre m\'{e}thode souffrait principalement
du manque de pr\'{e}cision des d\'{e}tecteurs d'objets. 
