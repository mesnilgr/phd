\chapter{Pr\'{e}ambule au Premier Article }

\section{D\'{e}tails de l'article}

{\bf Unsupervised and Transfer Learning Challenge: a Deep Learning Approach}
Gr\'{e}goire Mesnil, Yann Dauphin, Xavier Glorot, Salah Rifai, Yoshua Bengio,
Ian Goodfellow, Erick Lavoie, Xavier Muller, Guillaume Desjardins, David
Warde-Farley, Pascal Vincent, Aaron Courville and James Bergstra. in: Journal
of Machine Learning Research: Proceedings of the Unsupervised and Transfer
Learning Challenge and workshop, pages 97-110, 2012}  

{\it Contribution Personnelle} Cette comp\'{e}tition faisait partie du travail
pratique d'un cours de Yoshua Bengio. Yann Dauphin, Xavier Glorot, Salah Rifai
et moi-même avons décidé de former un noyau dur pour remporter la compétition
(plus de 800 entrées à nous 4 sur une courte période avec une centaine de compétiteurs).
J'ai proposé plusieurs solutions décisives pour cette victoire comme la
transductive-PCA, et plusieurs de mes entrées nous ont permis de remporter la
première place. Une partie de mon code a été mis à disposition des autres
étudiants sous forme de tutorial afin de leur permettre de participer et
dèutilser notre cluster de calcul. Par la suite, J'ai dirigé la rédaction de
l'article en collaboration avec tous les co-auteurs et préparé la présentation
donnée par Yann Dauphin lors de la conférence ICML 2012. Le papier a remporté
le PASCAL 2 UTLC best paper award. 

\section{Contexte}

À cette époque, l'apprentissage de représentations non supervisé commence à
prendre son essor mais de nombreux scientifiques sont encore suspicieux
vis-à-vis du "Deep". Cette victoire a contribué a montré que ces techniques
pouvaient rivaliser avec d'auters méthodes à l'état de l'art comme les SVM.

\section{Contributions}

Plusieurs des techniques présentées dans ce papier ont par la suite obtenu des
publications dans des journaux ou des conférences internationales comme le S3C
TODO citeou le CAE cite TODO. Le pipeline présenté ici a aussi été utilisé pour
obtenir une meilleure représentation avec une réduction de dimensionnalité
conséquente dans le second article de cette dissertation. 

\section{R\'{e}cents d\'{e}veloppements}

Avec la récente mise en avant de l'apprentissage supervisé et de ces succès,
l'apprentissage non supervisé est un peu mis en retrait même si cela reste
définitivement le graal pour de nombreux chercheurs étant donné la quantité
massive de donnnées non labellées disponible via internet. Cela a néanmoins
permis de fédérer une équipe de recherche avec qui j'ai pu travailler sur
d'autres projets. Dans le futur, j'espère pouvoir perpétuer ces riches
collaborations.
