\chapter*{R\'{e}sum\'{e}\\~~~~~~~~\\~~~~~~~~~~~}

\vspace{-2.8cm}

Dans cette dissertation, nous présentons plusieurs techniques d'apprentissage
d'espaces sémantiques pour différents domaines, par exemple des mots et des
images, mais aussi à l'intersection de différents domaines.  On utilisera
différentes techniques d'apprentissage machine pour apprendre des
représentations avec des propriétés intrasèques intéressantes. Ces propriétés
peuvent aller d'un meilleur mixage (un espace où il est plus aisé de faire de
l'échantillonage), à une meilleure séparabilité linéaire (un espace où il est
plus facile de discrimer ave un hyperplan). L'espace sémantique est l'espace où
la représentation évolue. Cet espace est appelé sémantique si des entités
jugées similaires par un être humain, ont leur similarité préservée dans cet
espace.
\\

\vspace{-0.2cm}
La première publication présente un enchaînement de méthodes d'apprentissage
incluant de nombreuses techniques d'apprentissage non supervisé utilisées pour
gagner une compétition en 2011. La particularité de cette compétition était
qu'aucun apriori sur la structure des données ne pouvait être utilisé car les
différents ensembles de données (reconnaissance de caractères manuscrits,
actions humaines, traitement du langage naturel, données écologiques,
classification d'images) étaient rendus anonymes par des permutations
aléatoires. Le deuxième article présente une manière d'extraire de l'information
partir d'un contexte structuré ($177$ détecteurs d'objets à différentes
positions et échelles). On montrera que l'utilisation de la structure des
données combinée à un apprentissage non supervisé permet de réduire la
dimensionnalité d'un facteur de $97\%$ tout en améliorant les performances de
reconnaissance de scènes de $+5\%$ à $+11\%$ suivant l'ensemble de données.
\\

\vspace{-0.2cm}
Dans le troisième travail, on s'intéresse à la structure apprise par les
réseaux de neurones profonds utilisés dans les deux précédentes publications.
Plusieurs hypothèses sont présentées et testées expérimentalement montrant que
l'espace appris a de meilleures propriétés de mixage (explorer toutes les
différentes classes durant le processus d'échantillonage).
\\

\vspace{-0.2cm}
Pour la quatrième publication, on s'intéresse à résoudre un problème d'analyse
syntaxique sémantique avec des réseaux de neurones récurrents appris sur des
fenêtres de contexte de mots.  On observe une nette amélioration de l'état de
l'art sur plusieurs ensembles de données et d'intéressantes propriétés de
l'espace sémantique associé au langage.  Une recherche sur l'apprentissage d'un
espace sémantique à l'intersection des mots et des images est présenté dans
notre cinquième travail. On propose une façon d'effectuer de la recherche
d'image "augmentée" en apprenant un espace sémantique où une recherche d'image
contenant un objet retournerait aussi des images des parties de l'objet, par
exemple une recherche retournant des images de  "voiture" retournerait aussi
des images de "pare-brises", "coffres", "roues" en plus des images initiales.
 
\chapter*{Summary\\~~~~~~~~\\~~~~~~~~~~~\\~~~~~~~~~~}

\vspace{-3cm}

In this work, we focus on learning semantic spaces for multiple domains, but
also at the intersection of different domains. The semantic space is where the
learned representation lives. This space is called semantic if similar
entities from a human perspective have their similarity preserved in this
space.  We use different machine learning algorithms to learn representations
with interesting intrinsic properties. 
%These properties range from better
%mixing to better linear separability. 
\\

\vspace{-0.2cm} The first article presents a pipeline including many different
unsupervised learning techniques used to win the Unsupervised and Transfer
Learning Challenge in 2011. No prior knowledge on the structure of the data
could be used during this competition since the different datasets were
anonymous. The learned representations improved the performance of a weak
classifier compared to the raw data. It also transfered well to different
subsets of classes for the $5$ different domain datasets in this competition.
%arabic character recognition, CIFAR-10, Natural Language processing, human
%action recognition and ecological data.

In the second article, we present a pipeline widely inspired from the one above
but taking advantage of the structure of the data for a scene classification
problem. We present a way to learn from structured context, consisting of $177$
object detections at different poses and scales. We show that using the
structure of the data combined with the Contractive Auto-Encoder, an
unsupervised learning algorithm, 
%and Principal Component Analysis for preprocessing 
allows us to drastically reduce the dimensionality while improving
significantly on the scene recognition accuracy. 
%This pipeline is widely inspired
%from the one above used to win the competition except that, in addition, we are
%able to take advantage of the structure of the data.  
\\

\vspace{-0.2cm}
The third article focuses on the space structure learned by deep
representations. Our experiments use greedy layer-wise unsupervised training of
Contractive Auto-Encoders and Restricted Boltzmann Machines. Several hypothesis
are experimentally tested and show that abstract representation spaces have
better mixing properties. We conclude that performing the sampling procedure
from the representation space explores more of the different classes.
\\

\vspace{-0.2cm}
In the fourth article, we tackle a semantic parsing problem with several
Recurrent Neural Network architectures taking as input context windows of word
embeddings. We show an improvement over the state of the art previously
obtained with Conditional Random Fields on different datasets. Learning context
windows of word embeddings leads to a
word semantic space where words with the same output class are
clustered together. Depending on the dataset, performing Viterbi decoding with
the probabilities of the model is crucial to obtain good performance.

In the fifth article, an investigation on learning a single semantic space at
the intersection of words and images is presented. As we lack of labeled images
of objects and their parts, we use the Word-Net part-of relationship to obtain
a training proxy. Through the semantic space intersecting words and images
domains, we propose a way to perform "augmented search" where a search on an
image containing an object would also return images of the object's parts.
%, e.g.
%searching for "car" images would return "windshield", "trunk" or "wheel" images
%in addition to car images.

\chapter*{Articles}

\begin{enumerate}

\item Grégoire Mesnil, Yann Dauphin, Xavier Glorot, Salah Rifai, Yoshua Bengio,
et al. {\bf Unsupervised and Transfer Learning Challenge: a Deep Learning
Approach} (2012) Journal of Machine Learning Workshop and Conference Papers.

\item Grégoire Mesnil, Salah Rifai, Antoine Bordes, Xavier Glorot, Yoshua Bengio and Pascal Vincent
{\bf Unsupervised Learning of Object Detections Semantics for Scene Categorization}
(2015) Lecture Notes in Advances in Intelligent and Soft Computing, Springer-Verlag.

\item Yoshua Bengio\footnote{indicates equal contribution}, Grégoire Mesnil$^{1}$, Yann Dauphin and Salah Rifai
{\bf Better Mixing via Deep Representations} (2013) International
Conference on Machine Learning

\item Grégoire Mesnil, Yann Dauphin, Kaisheng Yao, Yoshua Bengio, Li Deng, Dilek Hakkani-Tur, Xiaodong He, Larry Heck, Gokhan Tur, Dong Yu and Geoffrey Zweig {\bf Using Recurrent Neural Networks for Slot Filling in Spoken Language Understanding} (2015) IEEE Transactions on Audio, Signal and Language Processing

\item Grégoire Mesnil, Antoine Bordes, Jason Weston,
Gal Chechik and Yoshua Bengio, {\bf Learning Semantic Representations Of
Objects and Their Parts} (2013) Machine Learning Journal

\end{enumerate}
