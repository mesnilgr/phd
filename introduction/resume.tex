\chapter*{R\'{e}sum\'{e}}

lala

\chapter*{Summary}

In this work, we focus on learning semantic spaces for different domains, e.g.
words and images, but also at the intersection of different domains. We use
different machine learning algorithms to learn representations with interesting
intrinsic properties. These properties range from better mixing (a space where
it's easier to perform sampling), to better linear separability (a space where
it's easier to discriminate with an hyperplane). The semantic space is where
the learned representation lives. This space is called semantic if "similar"
entities from a human perspective have their similarity preserved in this
space.\\

The first article presents a pipeline including many different unsupervised
learning techniques used to win a challenge in 2011. No prior on the structure
of the data could be used during this competition since the different datasets
(arabic manuscript recognition, human actions, NLP, ecology data, CIFAR10) were
anonymous. In the second article, we present a way to learn from structured
context ($177$ object detections at different poses and scales) for performing
scene categorization. We show that using the structure of the data combined
with unsupervised learning algorithms allow us to reduce the dimensionality by
a factor of $97\%$ while improving the scene recognition $+5\%$ to $+11\%$
depending on the dataset.\\

In the third article, we focus on the space structure learned by deep
representations. Several hypothesis are experimentally tested and show that
abstract representation spaces have better mixing properties (exploring all the
different classes during the sampling procedure).\\

In the fourth article, we tackle a semantic parsing problem with Recurrent
Neural Network architectures and context windows of word embeddings. We show an
improvement over the state of the art on several datasets and interesting
semantic properties in the learned word "embedding" space.  An investigation on
learning a single semantic space at the intersection of words and images is
presented in the fifth article. We propose a way to perform "augmented search"
by learning a semantic space where a search on an image containing an object
would also return images of the object's parts, e.g. searching for "car" images
would return "windshield", "trunk" or "wheel" images in addition to car images.
