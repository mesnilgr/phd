\chapter*{Remerciements}
\vspace{-0.5cm}
Pour commencer, je tiens à remercier Yoshua pour son ouverture à toujours
discuter de nouvelles idées et sa rapidité d'exécution pour les mettre sous
presse. Le LISA est un environnement de recherche d'exception, un des rares
endroits où l'on peut voir un laboratoire surpasser la recherche industrielle
avec des moyens académiques. J'aimerais aussi remercier mes co-directeurs Pascal
et Alain pour m'avoir fait confiance et octroyé une grande liberté dans ma
recherche.
\\

\vspace{-0.1cm}
Toute l'équipe du GAMME, j'y ai recontré des collègues et amis qui me sont
chers: mon wingman faucon bassiste préféré Yann, notre futur iron-man national
Xavier. Mais aussi toutes les bouteilles de champagne et l'écumage de la rue
Laurier en mode stealth et beurre argentin avec Salah le stoïcien et Xavier
notre père à tous.  Je leur souhaite à tous une réussite à la hauteur de leurs
ambitions.
\\

\vspace{-0.1cm}
Les chercheurs talentueux que j'ai pu rencontrer en industrie dans l'équipe de
Microsoft Research à Redmond, notamment Xiaodong et Li. Puis à Facebook
Articial Intelligence Research en Californie avec Antoine, Marc'Aurelio, Tomas
et Jason.
\\

\vspace{-0.1cm}
Parmi mes racines françaises, Simon qui est passé devant dans la forêt
aléatoire, ma gang du Cotentin, avec Alban le californien et Florence la femme
fatale en tête. Toujours, la famille avec mes parents sans qui je ne serais sans
doute pas là, et mes deux petites soeurs, Margaux et Clémence, déjà rendues bien grandes.
\\

\vspace{-0.1cm}
Mes belles rencontres montréalaises avec Michelle et Séb, pour les vibrations
dans le placard et les colons; Baptiste, pour les soirées Michelle Drucker
dans l'Unité, Marlène, Val, Marie-Anne, Thibault les meilleur(e)s
colocataires du monde, Maude, Pat et Miki, mes photographes préférés, et la
gang du ski de la Fondation des Aveugles. Et bien sûr mes recontres
califoniennes, avec Bob et ma très chère Amina.
\\

\vspace{-0.1cm}
J'aimerais aussi remercier ceux qui ont rendu mon doctorat possible
au travers de leurs financements et de leurs moyens de calcul: le DIRO, l'ANR,
la FESP et le Consul de France; NSERC, Calcul Québec, Compute Canada et CIFAR
et toute l'équipe derrière Theano, Frédéric en tête. 
\\

\vspace{-0.1cm}
Pour terminer, j'aimerais remercier tout le monde, j'espère ainsi que je n'ai
oublié personne.
