\chapter{Pr\'{e}ambule au Cinqui\`{e}me Article }

\section{D\'{e}tails de l'article}

{\bf Learning Semantic Representations of Objects and their Parts} Grégoire
Mesnil, Antoine Bordes, Jason Weston, Gal Chechik and Yoshua Bengio, {\it
Machine Learning Journal: Special Issue on Learning Semantics}, 2013

{\it Contribution personnelle} Le projet a débuté suite à une visite d'Antoine
Bordes dans les bureaux de Google. Il m'a ensuite proposé de prendre part au
projet. L'idée originale est d'Antoine, Jason Weston s'est ensuite chargé de
nous fournir des représentations d'images utilisées en industrie par Google à
cette époque. J'ai réalisé l'intégralité des exprériences et participé à la
rédaction avec la collaboration des co-auteurs. La partie expérience s'est
révélée être très intéressante car elle consistait à effectuer une tâche
d'apprentissage avec un très grand nombre de données, plusieurs millions de
couples.  Nous remercions les reviewers anonymes du Machine Learning Journal
qui ont aussi contribué à la qualité de l'article au travers de leurs
questionnements.

\section{Contexte}

Au début de ces recherches, l'idée de partager des espaces sémantiques entre
plusieurs domaines est assez neuve \citep{image-wsabie}. Cependant, l'idée
d'utiliser des embeddings pour modéliser le langage a déjà fait son chemin
\citep{bengio:2003}, ces méthodes obtiennent même d'excellentes performances
sur diverses tâches de traitement du langage naturel \citep{collobert:2011b}.
L'originalité de notre travail est d'effectuer un transfert d'apprentissage au
travers d'un espace sémantique.

\section{Contributions}

J'ai eu la chance de pouvoir présenter cet article à l'oral à l'UTC de
Compiègne devant l'équipe recherche ainsi qu'au GDR Information Signal Image
Vision à Paris devant $~70$ personnes. Nous espérons que l'idée d'utiliser des
espaces sémantiques pour effectuer du transfert d'apprentissage viendront
enrichir le paysage de la recherche actuelle.

\section{R\'{e}cents d\'{e}veloppements}

On peut observer sur Google Image dans le système de production des suggestions
de navigation pour d'autres mots clés de recherche. Par exemple, une recherche
initiale pour "voiture" va ensuite suggérer "voiture sport", "f1", "voiture
jaune" qui sont des prémisses de recherche augmentée même si cela reste au
niveau sémantique et n'atteint pas l'ensemble d'images retournées.  Autrement,
le système Devise \citep{samy-extreme} à base d'espace sémantique est
actuellement utilisé par Google pour son moteur de recherche d'images.

