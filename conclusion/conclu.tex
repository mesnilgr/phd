
\chapter{Conclusion Générale}

L'apprentissage de réseaux de neurones profonds a évolué très rapidement au
cours de la réalisation de ce doctorat. Ces changements sont reflétés dans nos
travaux qui ont suivi l'évolution riche de la recherche en architectures
profondes des dernières années.

Nous avons commencé par présenter de nombreuses techniques d'apprentissage non
supervisé, puis comment les mettre en oeuvre pour gagner une compétition de
transfert d'apprentissage ou améliorer la performance d'un système de vision
tout en réduisant la dimensionnalité de l'espace d'entrée original.  Nos
intuitions concernant l'espace sémantique des représentations apprises sont
résumées dans nos recherches sur l'amélioration du mixage lors du processus
d'échantillonage effectué dans les couches profondes, les plus abstraites.

L'utilisation d'architectures prometteuses telles que les réseaux de neurones
récurrents a été la base d'une transition vers l'apprentissage supervisé
d'espaces sémantiques associés au langage. À l'heure actuelle, tous les récents succès de
l'industrie résident dans l'utilisation d'ensemble de données de très grande échelle.
C'est la nature de nos derniers travaux qui se concentrent sur le transfert
d'apprentissage au travers d'espace sémantiques multi-domaines avec un grand
ensemble de données (plusieurs millions d'exemples).

Dans le futur, nous espérons poursuivre nos travaux de recherche pour apprendre
des espaces sémantiques avec une structure plus riche et un nombre de domaine
arbitraire.

\chapter*{Bibliographie Personnelle}

\begin{description}
\item[2015] \hfill \\
\begin{description}
\item \uline{Grégoire Mesnil}, Yann Dauphin, Kaisheng Yao, Yoshua Bengio, Li Deng, Dilek Hakkani-Tur, Xiaodong He, Larry Heck, Gokhan Tur, Dong Yu and Geoffrey Zweig {\bf Using Recurrent Neural Networks for Slot Filling in Spoken Language Understanding} (2015) IEEE Transactions on Audio, Signal and Language Processing
\vspace{0.1cm}
\item \uline{Grégoire Mesnil}, Salah Rifai, Antoine Bordes, Xavier Glorot, Yoshua Bengio and Pascal Vincent
{\bf Unsupervised Learning of Object Detections Semantics for Scene Categorization}
(2015) Lecture Notes in Advances in Intelligent and Soft Computing, Springer-Verlag
\vspace{0.1cm}
\item \uline{Grégoire Mesnil}, Tomas Mikolov, Marc'Aurelio Ranzato and Yoshua Bengio {\bf Ensemble of Generative and Discriminative Techniques for Sentiment Analysis of Movie Reviews} (2015) International Conference on Learning Representations (submitted in the workshop track)
\vspace{0.1cm}
\end{description}

\item[2014] \hfill \\
\begin{description}
\item Yelong Shen, Xiaodong He, Jianfeng Gao, Li Deng and \uline{Grégoire Mesnil}
{\bf  Learning Semantic Representations Using Convolutional Neural Networks for Web Search}
(2014) World Wide Web Conference, poster.
\vspace{0.1cm}
\item Yelong Shen, Xiaodong He, Jianfeng Gao, Li Deng and \uline{Grégoire Mesnil}
{\bf  A Convolutional Neural Network based Latent Semantic Model for Web Search}
(2014) CIKM Conference on Knowledge and Information Management.
\vspace{0.1cm}
\end{description}

\item[2013] \hfill \\
\begin{description}
\item \uline{Grégoire Mesnil}, Xiaodong He, Li Deng and Yoshua Bengio
{\bf Investigation of Recurrent-Neural-Network Architectures and Learning Methods for Spoken Language Understanding}
(2013) Interspeech
\vspace{0.1cm}
\item Yoshua Bengio, \uline{Grégoire Mesnil}, Yann Dauphin and Salah Rifai
{\bf Better Mixing via Deep Representations} (2013) International
Conference on Machine Learning
\vspace{0.1cm}
\item \uline{Grégoire Mesnil}, Antoine Bordes, Jason Weston,
Gal Chechik and Yoshua Bengio, {\bf Learning Semantic Representations Of
Objects and Their Parts} (2013) Machine Learning Journal
\vspace{0.1cm}
\item \uline{Grégoire Mesnil}, Salah Rifai, Xavier Glorot, Antoine Bordes,
Yoshua Bengio and Pascal Vincent, {\bf Unsupervised and Transfer Learning under Uncertainty:
from Object Detections to Scene Categorization} (2013) International Conference on Pattern
Recognition Applications and Methods (oral)
\vspace{0.1cm}
\end{description}

\item[2012] \hfill \\
\begin{description}
\item \uline{Grégoire Mesnil}, Salah Rifai, Yann Dauphin,  Yoshua Bengio and
Pascal Vincent, {\bf Surfing on the Manifold } (2012) Learning Workshop
(poster), Snowbird, Utah, U.S.A.
\vspace{0.1cm}
\item
\uline{Grégoire Mesnil}, Yann Dauphin, Xavier Glorot, Salah Rifai, Yoshua Bengio, et al. {\bf Unsupervised and Transfer Learning Challenge: a Deep Learning Approach} in Journal of Machine Learning Workshop and Conference Papers (JMLR WCP 2012).
\vspace{0.1cm}
\end{description}

\item[2011] \hfill \\
\begin{description}
\item \uline{Grégoire Mesnil}, Salah Rifai, Xavier Glorot, Antoine Bordes, Pascal Vincent and Yoshua Bengio, {\bf Exploiting context-based Information for Scene Categorization} (2011) at NIPS Workshop on Learning Semantics (poster)
\vspace{0.1cm}
\item
Salah Rifai, \uline{Grégoire Mesnil}, Pascal Vincent, Xavier Muller, Yoshua Bengio, Yann Dauphin and Xavier Glorot, {\bf Higher Order Contractive Auto-Encoder} in Proceedings of the European Conference on Machine Learning (ECML 2011)
\vspace{0.1cm}
\item
Salah Rifai, Xavier Muller, \uline{Grégoire Mesnil}, Yoshua Bengio and Pascal Vincent, {\bf Learning Invariant features through local space contraction} in Technical Report 1360. Département d'informatique et de recherche opérationnelle. Université de Montréal 2011.
\end{description}

\end{description}


